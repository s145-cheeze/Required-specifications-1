%%This is a very basic article template.
%%There is just one section and two subsections.
\documentclass[12pt]{jsarticle}
\usepackage[top=15truemm,bottom=15truemm,left=13truemm,right=13truemm]{geometry}
\usepackage{amsmath}
\usepackage{amssymb}
\usepackage{ascmac}
\usepackage{fancybox}
\usepackage[usenames]{color}
\usepackage{plext}
\usepackage{comment}
\usepackage{ulem}

%%赤文字と赤下線
\newcommand{\ruline}[1]{\textcolor{red}{\underline{\textcolor{black}{#1}}}}
\newcommand{\red}[1]{\textcolor{red}{#1}}
%%改行用コマンド
\newcommand{\ret}{\\\quad}
\newcommand{\tret}[1]{\tag{#1}\\}
\newcommand{\ntret}{\notag\\}


\renewcommand{\thesection}{第\arabic{section}章}

\title{「TrimerX」要求仕様書}
\author{チーム「ちぃーず」\\05伊藤いちご\ 18金子航\ 24米華真典}
\date{}


\begin{document}
\maketitle
\newpage

\tableofcontents

\newpage

\setcounter{page}{1}

\section{概要}
本ソフト「TrimerX」は試験の答案やアンケートの集計をするための,画像のトリミングを主としたものです.トリミングしたものをまとめて見ることや答案の点数を入力・集計して平均点を計算することが可能です.

\newpage

\section{動作・開発環境}
\begin{itemize}
    \item Python3.6.4
    \item PyQt5.6.2
    \item OpenCV
\end{itemize}

\newpage


\section{ソフトウェアの構成}
本ソフトのトリミング機能は大きく分けて2種類になります.

用途に合わせてその機能を使い分けることによってデータの集計を円滑に行うことが可能です.
\begin{enumerate}
    \item トリミング
    \item データビュー
    \item データ入力・集計
    \item 画像の読み込み
\end{enumerate}

\newpage

\section{各機能の詳細}

\subsection{トリミング}
基本的なトリミング2種類を行えます.

\begin{enumerate}
    \item 定型トリミング \\
    主にアンケート用紙などの,大量のデータに対して位置を設定し連続的にトリミングを行います.
    切り取ったデータは種類ごとに同じファイルにまとめられ「データビュー」にて閲覧が可能となります.
    \item 自由トリミング
    主に自由記述型のテキストや解答用紙などに対して機能します.
    各解答に対しておおよその位置をマウスポインターによって指定することで,指定した位置に対して適切なトリミングを行います.
    一枚のプリントに対して連続的に位置を指定することができ,指定した順番によってファイル名も設定されます.

\end{enumerate}

\subsection{データビュー}
データが種類ごとに分別されおり,閲覧することが可能です.
\subsection{データ入力・集計}
テストの解答に対して,問いごとに点数の入力を行えます.入力したデータはcsvファイルに出力され,平均点,合計点などの集計が可能です.
\subsection{画像の読み込み}
画像ファイル(.jpg)の読み込みはファイル毎に行われ,存在しているファイルによって属性が割り当てられます.

\newpage
\end{document}
